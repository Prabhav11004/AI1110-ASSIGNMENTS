%%%%%%%%%%%%%%%%%%%%%%%%%%%%%%%%%%%%%%%%%%%%%%%%%%%%%%%%%%%%%%%
%
% Welcome to Overleaf --- just edit your LaTeX on the left,
% and we'll compile it for you on the right. If you open the
% 'Share' menu, you can invite other users to edit at the same
% time. See www.overleaf.com/learn for more info. Enjoy!
%
%%%%%%%%%%%%%%%%%%%%%%%%%%%%%%%%%%%%%%%%%%%%%%%%%%%%%%%%%%%%%%%


% Inbuilt themes in beamer
\documentclass{beamer}

% Theme choice:
\usetheme{CambridgeUS}

% Title page details: 
\title{Assignment 14} 
\author{PRABHAV SINGH (BT21BTECH11004)}
\date{17 June 2022}
\logo{\large \LaTeX{}}

\begin{document}
	
	% Title page frame
	\begin{frame}
		\titlepage 
	\end{frame}
	
	% Remove logo from the next slides
	\logo{}
	
	
	% Outline frame
	\begin{frame}{Outline}
		\tableofcontents
	\end{frame}
	
	% Lists frame
	\section{Question}
	\begin{frame}{Question}
	Show that the weighted sample spectrum $ S_{c}(\omega)  = \dfrac{1}{2T}{|\int_{-T}^{T}c(t)x(t)e^{-j{\omega}t}|}^2 $ of process $ x(t) $ is the Fourier Transformation of the function\\
	\begin{equation}
		R_{c}(\tau) =\int_{-T+|\tau/2|}^{T-|\tau/2|} c\bigg(t+\dfrac{{\tau}}{2}\bigg)c\bigg(t-\dfrac{{\tau}}{2}\bigg)x\bigg(t+\dfrac{{\tau}}{2}\bigg)x\bigg(t-\dfrac{{\tau}}{2}\bigg)\;dt
	\end{equation}
	\end{frame}
	
	
	% Blocks frame
	\section{Solution}
	\begin{frame}{Solution}
	The Function 
	\begin{equation}
		X_{c}(\omega)=\int_{-T}^{T}c(t)x(t)e^{-j{\omega}t}\;dt
	\end{equation}
	is the Fourier Transformation of the product $ c(t)x_{T}(t) $; where $  x_{T}(t)=\left\{
	\begin{array}{ll}
		1 & \quad |t| < T  \\
		0 & \quad |t| > T 
	\end{array}
	\right. $\\
	Hence , the function $ 2T.S_{T}(\omega) = 	{|X_{c}(\omega)|}^2 $ is the Fourier Transformation of \\
	\begin{align}
		f(t)= {c(t)x_{T}(t)}.c(-t)x_{T}(-t)  \\ 
		=\int_{-T+|\tau/2|}^{T-|\tau/2|} c\bigg(t+\dfrac{{\tau}}{2}\bigg)c\bigg(t-\dfrac{{\tau}}{2}\bigg)x\bigg(t+\dfrac{{\tau}}{2}\bigg)x\bigg(t-\dfrac{{\tau}}{2}\bigg)\;dt 
	\end{align}
	\end{frame} 
	

\end{document}