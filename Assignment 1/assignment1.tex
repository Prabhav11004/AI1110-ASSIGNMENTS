\documentclass[journal,12pt,twocolumn]{IEEEtran}


\usepackage{amsmath,amssymb,amsthm}



\begin{document}
	\vspace{3cm}
	\title{ASSIGNMENT 1}
	\author{Prabhav Singh- BT21BTECH11004}
	
	\maketitle
	\textbf{PROBLEM 9(b)}:-Using Properties of proportion solve for x, given\\
	
	
	\hspace*{2cm}$ 	\dfrac{\sqrt{5x}+\sqrt{2x-6}}{\sqrt{5x}-\sqrt{2x-6}} =4 $
	
	
	\medskip
	
	
	\textbf{SOLUTION}:-\\
	Using Componendo and Dividendo rule that is if $  \frac{a}{b} =\frac{c}{d} \implies \frac{a+b}{a-b} =\frac{c+d}{c-d}; $ on the given expression\\
	
	 \begin{equation}
	 		\dfrac{\sqrt{5x}+\sqrt{2x-6}}{\sqrt{5x}-\sqrt{2x-6}} =\dfrac{4}{1} 
	 \end{equation}
	
	\begin{align}
		\dfrac{\sqrt{5x}+\sqrt{2x-6}+\sqrt{5x}-\sqrt{2x-6}}{\sqrt{5x}+\sqrt{2x-6}-\sqrt{5x}+\sqrt{2x-6}} =\dfrac{4+1}{4-1} \\ 
		\dfrac{2\sqrt{5x}}{2\sqrt{2x-6}} =\dfrac{5}{3} \\
		3(\sqrt{5x})=5(\sqrt{2x-6}) \\
		9\times5x=5\times5\times(2x-6) \\
		\ 9x=10x-30 \\
		\implies \boxed{ x=30} 
	\end{align}
	
	
	
\end{document}