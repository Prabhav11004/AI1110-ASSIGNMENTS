%%%%%%%%%%%%%%%%%%%%%%%%%%%%%%%%%%%%%%%%%%%%%%%%%%%%%%%%%%%%%%%
%
% Welcome to Overleaf --- just edit your LaTeX on the left,
% and we'll compile it for you on the right. If you open the
% 'Share' menu, you can invite other users to edit at the same
% time. See www.overleaf.com/learn for more info. Enjoy!
%
%%%%%%%%%%%%%%%%%%%%%%%%%%%%%%%%%%%%%%%%%%%%%%%%%%%%%%%%%%%%%%%


% Inbuilt themes in beamer
\documentclass{beamer}

% Theme choice:
\usetheme{CambridgeUS}

% Title page details: 
\title{Assignment 13} 
\author{PRABHAV SINGH (BT21BTECH11004)}
\date{\today}
\logo{\large \LaTeX{}}

\begin{document}
	
	% Title page frame
	\begin{frame}
		\titlepage 
	\end{frame}
	
	% Remove logo from the next slides
	\logo{}
	
	
	% Outline frame
	\begin{frame}{Outline}
		\tableofcontents
	\end{frame}
	
	% Lists frame
	\section{Question}
	\begin{frame}{Question}
	Show that if $ {\phi} $ is a random variable with $  \phi(\lambda)=E\{e^{f{\lambda}{\phi}}\} $ and $  {\phi}(1) = {\phi}(2) = 0 $, then the 
	process $ x(t) = \cos(wt+\phi ) $ is WSS. Find $ E\{x(t)\} $ and $ R_{x}({\tau}) $ if $\phi$ is uniform in the interval $ (-\pi,\pi) $
	\end{frame}
	
	
	% Blocks frame
	\section{Solution}
	\begin{frame}{Solution}
		From \begin{equation}
			{\phi}(1) = {\phi}(2) = 0
		\end{equation}   we can conclude that \\
		\begin{align}
			E\{\cos{\phi}\}&=E\{\sin{\phi}\}=E\{\cos2{\phi}\}=E\{\sin2{\phi}\}=0\\
			\implies	 E\{x(t)\}&=\cos(wt)E\{\cos{\phi}\}-\sin(wt)E\{\sin{\phi}\}
		\end{align}
		And using the result \\
		\begin{equation}
			2 \cos [\omega(t +\tau) + \phi] \cos ({\omega}t + \phi) = \cos({\omega}{\tau}+\phi)+\cos(2{\omega}t +{\omega}{\tau} + 2{\phi}) 
		\end{equation}

\end{frame}

\begin{frame}
	\begin{equation}
		2R_{x}({\tau})=\cos({\omega}{\tau})
	\end{equation}
	If $ {\phi} $ is uniform in  $ (-\pi,\pi) $ , then\\
	\begin{equation}
		\phi(\lambda)=\dfrac{\sin({\pi}{\omega})}{{\pi}{\omega}}~ and~ {\phi}(1) = {\phi}(2) = 0
	\end{equation}

\end{frame}
\end{document}