
\documentclass[journal,12pt,twocolumn]{IEEEtran}
\usepackage{amsfonts}
\usepackage{amsmath}
\usepackage{amssymb}
\usepackage{bigints}
\DeclareMathOperator{\di}{d\!}

\begin{document}
	\title{\textbf{AI1110 Assignment 2} }
	\author{\textbf{Prabhav Singh}\\\textbf{BT21BTECH11004}\\ ICSE Grade 12 2019 paper}
	\maketitle
	
	{\section*{Question 13}}
	
	{\large \underline{Question}:\newline}
	
	Evaluate:
	\begin{equation}
		\int_{0}^{{\pi}} \dfrac{x \tan x}{\sec x + \tan x } \,\di x 
	\end{equation}\\
	
	{\large \underline{Solution}:}\\
	 	\begin{equation}
	 I=	\int_{0}^{{\pi}} \dfrac{x \tan x}{\sec x + \tan x } \,\di x 
	 \end{equation}\\
	\begin{align}
     \implies I &=	\int_{0}^{{\pi}} \dfrac{({\pi-x})\tan({\pi-x})}{\sec({\pi-x})  + \tan ({\pi-x}) } \,\di x   \\
	\implies I 	&= {\pi}\int_{0}^{{\pi}} \dfrac{\tan x}{\sec x  + \tan x } \,\di x  -I \\
	\implies 2I	&=  {\pi}\int_{0}^{{\pi}} \dfrac{\tan x}{\sec x  + \tan x } \,\di x \\
	        	&=  {\pi}\int_{0}^{{\pi}} \dfrac{\tan x}{\sec x  + \tan x } \times\dfrac{\sec x -\tan x}{\sec x -\tan x}\,\di x  \\
	        	&= {\pi}\int_{0}^{{\pi}} \tan x(\sec x-\tan x)\,\di x\\
	        	&=  {\pi}\int_{0}^{{\pi}}( \tan x\sec x-\tan^2 x)\,\di x\\
	        	&={\pi} \left[\sec x -\tan x +x \right]^{\pi}_{0}\\
	        	&= {\pi}(-1+{\pi}-1)\\
	\implies 2I &={\pi}({\pi}-2) \\
	\implies I  &= \dfrac{{\pi}({\pi}-2)}{2}      	
		\end{align}
	
\end{document}