%%%%%%%%%%%%%%%%%%%%%%%%%%%%%%%%%%%%%%%%%%%%%%%%%%%%%%%%%%%%%%%
%
% Welcome to Overleaf --- just edit your LaTeX on the left,
% and we'll compile it for you on the right. If you open the
% 'Share' menu, you can invite other users to edit at the same
% time. See www.overleaf.com/learn for more info. Enjoy!
%
%%%%%%%%%%%%%%%%%%%%%%%%%%%%%%%%%%%%%%%%%%%%%%%%%%%%%%%%%%%%%%%


% Inbuilt themes in beamer
\documentclass{beamer}

% Theme choice:
\usetheme{CambridgeUS}

% Title page details: 
\title{Assignment 11} 
\author{PRABHAV SINGH (BT21BTECH11004)}
\date{03 June 2022}
\logo{\large \LaTeX{}}

\begin{document}
	
	% Title page frame
	\begin{frame}
		\titlepage 
	\end{frame}
	
	% Remove logo from the next slides
	\logo{}
	
	
	% Outline frame
	\begin{frame}{Outline}
		\tableofcontents
	\end{frame}
	
	% Lists frame
	\section{Question}
	\begin{frame}{Question}
	We have shown that if $  X  $ is a random variable with distribution $ F(x) $, 
	then the random variable $ y = F(x) $ is uniform in the interval $  (0, 1) $. The following is a 
	generalization. \\
	Given $ n $ arbitrary random variables $  x_{i} $; we form the random variables 
	$ y_{1}=F(x_{1}) ,y_{2}=F(x_{2}|x_{1}), ..... , y_{n}=F(x_{n}|x_{n-1},....,x_1{})  $.
	We shall show that these random variables are independent and each is uniform in the 
	interval $ (0, 1) $. 
	\end{frame}
	
	
	% Blocks frame
	\section{Solution}
	\begin{frame}{Solution}
	The random variables $ y_{1} $ are functions of the random variables $ x_{i}$ as assumed in initial stage of the problem ,For $  0 \leq y \leq 1 $, the system  \\
	$ y_{1}=F(x_{1}) ,y_{2}=F(x_{2}|x_{1}), ..... , y_{n}=F(x_{n}|x_{n-1},....,x_1{})  $\\
	has a unique solution $ x_{1},x_{2},x_{3},...,x_{n} $ and its jacobian equals \\
	$ J= \begin{vmatrix} \frac{ {\partial } y_{1}}{{ \partial } x_{1}}& 0 &0 .....&0 \\ \frac{{ \partial } y_{2}}{{\partial}x_{1}} & \frac{{\partial}y_{2}}{{\partial}x_{2}} &0 .....&0 \\ .&.&.......&. \\ \frac{{\partial}y_{n}}{{\partial}x_{n}} & . &.. .....& \frac{{\partial}y_{1}}{{\partial}x_{1}} \end{vmatrix}  $\\
	This determinant is triangular; hence it equals the product of its diagonal elements \\
	$ \dfrac{{\partial } y_{k}}{{\partial}x_{k}}=f(x_{k}|x_{k-1},....,x_{1}) $\\
	But we know by definition of conditional probability,\\
	\end{frame}
\begin{frame}
 \begin{equation}
 	f(x_{1},.....,x_{k})=f(x_{k}|x_{k-1},....,x_{1})......f(x_{2}|x_{1}).f(x_{1})
 \end{equation}
 So we have \\
 \begin{equation}
 	f(y_{1},.....,y_{n})=\dfrac{f(x_{1},.....,x_{k})}{f(x_{n}|x_{n-1},....,x_{1})......f(x_{2}|x_{1}).f(x_{1})}=1
 \end{equation}
 Hence it is proved that in the $ n $-dimensional cube $  0 \leq y_{i} \leq 1 $, and 0 otherwise. 
\end{frame}
\end{document}