%%%%%%%%%%%%%%%%%%%%%%%%%%%%%%%%%%%%%%%%%%%%%%%%%%%%%%%%%%%%%%%
%
% Welcome to Overleaf --- just edit your LaTeX on the left,
% and we'll compile it for you on the right. If you open the
% 'Share' menu, you can invite other users to edit at the same
% time. See www.overleaf.com/learn for more info. Enjoy!
%
%%%%%%%%%%%%%%%%%%%%%%%%%%%%%%%%%%%%%%%%%%%%%%%%%%%%%%%%%%%%%%%


% Inbuilt themes in beamer
\documentclass{beamer}

% Theme choice:
\usetheme{CambridgeUS}

% Title page details: 
\title{Assignment 7} 
\author{PRABHAV SINGH (BT21BTECH11004)}
\date{27 May 2022}
\logo{\large \LaTeX{}}

\begin{document}
	
	% Title page frame
	\begin{frame}
		\titlepage 
	\end{frame}
	
	% Remove logo from the next slides
	\logo{}
	
	
	% Outline frame
	\begin{frame}{Outline}
		\tableofcontents
	\end{frame}
	
	% Lists frame
	\section{Question}
	\begin{frame}{Question}
	Suppose there are $  r  $ successes in $ n $ independent Bernoulli trials. Find the conditional probability of a success on the $ i $th trial.
		
	\end{frame}
	
	
	% Blocks frame
	\section{Solution}
	\begin{frame}{Solution}
	Let us assume events A and B such that :\\
	A= $ r $ successes in $ n $ Bernoulli trials \\
	B=success at the $ i $th Bernoulli trial\\
	C= $ r-1 $ successes in the remaining $ n-1 $ Bernoulli trials excluding 
	the $ i $th trial \\
	\begin{align}
		P(A) &= \binom{n}{r}p^{r}q^{n-r}\\
		P(B) &= p\\
		P(C) &= \binom{n-1}{r-1}p^{r-1}q^{n-r}
	\end{align}
	\end{frame}
	\begin{frame}
	So the conditional probability of a success on the $ i $th trial ,\\
	\begin{align}
		P({B|A}) &= \frac{P(AB)}{P(A)} \\
		&= \frac{P(BC)}{P(A)}\\
		&= \frac{P(B)P(C)}{P(A)}\\
		\implies P(B|A)	&= \frac{r}{n}
	\end{align}
	\end{frame}
	
\end{document}