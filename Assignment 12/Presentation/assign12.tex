%%%%%%%%%%%%%%%%%%%%%%%%%%%%%%%%%%%%%%%%%%%%%%%%%%%%%%%%%%%%%%%
%
% Welcome to Overleaf --- just edit your LaTeX on the left,
% and we'll compile it for you on the right. If you open the
% 'Share' menu, you can invite other users to edit at the same
% time. See www.overleaf.com/learn for more info. Enjoy!
%
%%%%%%%%%%%%%%%%%%%%%%%%%%%%%%%%%%%%%%%%%%%%%%%%%%%%%%%%%%%%%%%


% Inbuilt themes in beamer
\documentclass{beamer}

% Theme choice:
\usetheme{CambridgeUS}

% Title page details: 
\title{Assignment 12} 
\author{PRABHAV SINGH (BT21BTECH11004)}
\date{06 June 2022}
\logo{\large \LaTeX{}}

\begin{document}
	
	% Title page frame
	\begin{frame}
		\titlepage 
	\end{frame}
	
	% Remove logo from the next slides
	\logo{}
	
	
	% Outline frame
	\begin{frame}{Outline}
		\tableofcontents
	\end{frame}
	
	% Lists frame
	\section{Question}
	\begin{frame}{Question}
	Suppose that the time between arrvals of patients in a dentist's office constitutes samples of 
	a random variable X with density $ {\theta}e^{-{\theta}x}U(x) $. The 40th patient arrived 4 hours after the first. 
	Find the 0.95 confidence interval of the mean arrival time $ {\eta}=\dfrac{1}{{\theta}} $
	
	\end{frame}
	
	
	% Blocks frame
	\section{Solution}
	\begin{frame}{Solution}
	The time of arrival of the 40th patient is the sum $ x_{1} +x_{2} +..... + x_{n} $ of 
	$ n=39 $ RVs with exponential distribution.\\
	\medskip
	We can estimate the mean 
	$ {\eta}=\dfrac{1}{{\theta}} $ of $  x $ in terms of its sample mean $  X=\dfrac{240}{39}=6.15  $
	minutes using Normal Approximation method.  \\
	\medskip
	Taking {$\lambda$}={$\eta$} and $ z_{0.975} $/{$\sqrt{39}$}=0.315\\
	\end{frame}
\begin{frame}
	\begin{align}
		P\left\{\dfrac{\overline{x}}{1.315}\ < {\eta} < \dfrac{\overline{x}}{0.685}\right\}= 0.95 \\
		\implies 4.68minutes  < {\eta} < 8.98minutes
	\end{align}
\end{frame}
\end{document}